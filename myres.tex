\documentclass[margin, centered]{res}
\topmargin=-0.5in
\oddsidemargin -.5in
\evensidemargin -.5in
\textwidth=6.5in
\itemsep=0in
\parsep=0in
\newsectionwidth{1in}
%\usepackage{sectsty}
\usepackage[pdftex]{graphicx}
\usepackage{enumitem}
\usepackage{wrapfig}
\usepackage{helvet}
\usepackage[colorlinks = true,
            linkcolor = blue,
            urlcolor  = blue,
            citecolor = blue,
            anchorcolor = blue]{hyperref}
\setlength{\textwidth}{6.5in} % Text width of the document
\setlength{\textheight}{720pt}


\usepackage{booktabs}

\usepackage{graphicx}
\usepackage{array}
\begin{document}
\begin{center}
    \hspace{-\hoffset}
    \huge\bf{\href{}{Harshavardhan Unnibhavi}}
\end{center}
\vspace{-4mm}
\moveleft\hoffset\vbox{\hrule width 19cm height 0.5pt}
\vspace{-7mm}

\begin{center}
    \hspace{-\hoffset}
    ~\textbullet~\href{mailto:harshanavkis@gmail.com}{harshanavkis@gmail.com}
    ~\textbullet~\href{https://github.com/harshanavkis}{https://github.com/harshanavkis}
\end{center}
\vspace{-7mm}

\begin{resume}

\section{\textbf{Education}}

\textbf{Indian Institute of Technology Dhanbad,Jharkhand,India} \hfill\textit{ July 2015 - present}\\
B.Tech in Electronics and Communication Engineering, Minor in Computer Science\\

\section{\textbf{Research Interests}}
\begin{itemize}
\item Deep Learning, Programming Languages, Software Engineering, Operating Systems
\end{itemize}

\section{\textbf{Internship Experience}}
\begin{itemize}
\item \textbf{Research Intern}, \textit{Simon Fraser University, Burnaby, BC, Canada}
\hfill\textit{May 2018- October 2018}\\
~\textbullet~ Worked on Deep Feature Compression for Collaborative Intelligence\\
~\textbullet~ Developed a simulator using the Keras framework to simulate the transmission of deep features.\\
~\textbullet~ The simulator can be found at, \href{https://github.com/SFU-Multimedia-Lab/DFTS}{DFTS}.
\end{itemize}
\begin{itemize}
\item \textbf{AI and Reasoning Engine Intern}, \href{https://www.voyazer.net/}{ZapMyTrip Travel Solutions}\hfill\textit{November 2017- January 2018}\\
~\textbullet~Responsible for integrating NLP algorithms to extract contextual and factual information from travel reviews, news articles, into their application.
\end{itemize}

\section{\textbf{Publications}}
\begin{itemize}
\item H. Unnibhavi, H. Choi, S. R. Alvar, and I. V. Bajic, "DFTS: Deep Feature Transmission Simulator," IEEE Multimedia Signal Processing Workshop (MMSP), Vancouver, BC, Aug. 2018
\end{itemize}

\section{\textbf{Academic projects}}
\begin{itemize}
\item \textbf{Network management in optical Software Defined Networks}\hfill\textit{August 2018 - Present}\\
~\textbullet~ Physical layer impairments can affect routing algorithms when the transmission medium is an optical fiber.\\
~\textbullet~ This project aims to devise novel control and data plane algorithms that can supplement existing shortest path routing algorithms by taking into consideration the effects of the physical layer.\\
\end{itemize}

\begin{itemize}
\item \textbf{End-To-End Text to Speech Model}\hfill\textit{October 2017- December 2017}\\
~\textbullet~ Design and Implementation of a Text to Speech model in PyTorch using Deep Learning.\\
~\textbullet~ It was designed for the Hindi language.
\end{itemize}

\section{\textbf{Self projects: refer to the github link for code}}
\begin{itemize}
\item \textbf{Kaleidoscope compiler}\hfill\textit{November 2018 - Present}\\
~\textbullet~ Developing a compiler in Haskell using LLVM, for kaleidoscope.\\
\end{itemize}

\begin{itemize}
\item \textbf{hash}\hfill\textit{October 2018}\\
~\textbullet~ A unix like shell written in C.\\
~\textbullet~ Concepts from process creation, pipes etc were used in its implementation.\\
\end{itemize}

\begin{itemize}
\item \textbf{halloc}\hfill\textit{November 2018}\\
~\textbullet~ A memory allocation library written in C.\\
~\textbullet~ Malloc-like and free-like API has been provided.\\
\end{itemize}

\begin{itemize}
\item \textbf{xv6 hack}\hfill\textit{October 2018- present}\\
~\textbullet~ The project includes modifying the xv6 kernel to obtain the desired characteristics.\\
~\textbullet~ Lottery scheduling(similar to Amazon EC2 instances) was integrated into the kernel.\\
~\textbullet~ Added the ability for the OS to throw an exception when a null pointer is dereferenced.\\
~\textbullet~ Added the ability to change the protection levels of some pages in the process's address page.\\
\end{itemize}

\begin{itemize}
\item \textbf{Machine Learning projects}\\
~\textbullet~ Easytorch: a graphical interface to PyTorch, to allow for rapid development and prototyping of models through a drag and drop interface. It is currently being developed.\\
~\textbullet~ Cats Vs Dogs: Kaggle competition, where an accuracy of 87.5\% was achieved on the test set, by designing a 6 layer convolutional network in TensorFlow.\\
~\textbullet~ Sentiment Analysis on Rotten Tomatoes dataset: Used word2vec embeddings and a random forest classifier to develop a sentiment classifier.\\
\end{itemize}

\begin{itemize}
\item \textbf{Robotics projects}\\
~\textbullet~ Robot controlled by an AVR ATMega8 Microcontroller.\\
~\textbullet~Built a Line follower,Edge and Wall Avoider bot.\\
~\textbullet~Built a GSM controlled bot using Dual Tone Multiple Frequency(DTMF) signalling.\\
\end{itemize}

\section{\textbf{Technical Skills}}
\begin{tabular}{{l}ccccc}

\textbf{Programming languages}&\multicolumn{3}{l}{C,Python,Java,C++,x86 Assembly, Haskell}\\ 
\textbf{Software Skills}&\multicolumn{3}{l}{MATLAB, AVR Studio, Git,\LaTeX}\\
\textbf{Operating Systems}&\multicolumn{3}{l}{Windows,Linux}\\
\textbf{Tools and Frameworks}&\multicolumn{3}{l}{TensorFlow, PyTorch, Keras, Qemu, LLVM}\\
\textbf{Hardware Skills}&\multicolumn{3}{l}{Arduino, AVR ATMega8, 8085, TMS320C31 DSK, 31 DSK}
\end{tabular}

\section{\textbf{Relevant Courses}}
\begin{itemize}
\item Signals and Systems,Digital Signal Processing, Microprocessors, Computer Networks, Computer Architecture, Operating Systems(current)
\item Data Structures,Algorithm Design and Analysis,Linear Algebra,Multivariable Calculus,Vector Calculus,Numerical and Statistical methods
\item Machine Learning(by Andrew Ng,Coursera),cs231n(Fei Fei Li and Andrej Karpathy) 
\end{itemize}


\section{\textbf{Selected Achievements}}
\begin{itemize}
\item Was awarded the Mitacs Globalink fellowship.
\item Was selected for the KVPY scholarship program from about 150,000 students, conducted by the Indian Institute of Science, Bangalore and the Department of Science and Technology, Government of India. I was declared among the top 1\% after a rigorous examination and an interview.(Since 2015)
\item Was among the top 1000, selected from a pool of 500,000 students all over the country after clearing the National Talent Search Examination, which is a national-level scholarship program in India.(Since 2013)
\end{itemize}


\end{resume}
\end{document}