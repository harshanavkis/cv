\documentclass[margin, centered]{res}
\topmargin=-0.5in
\oddsidemargin -.5in
\evensidemargin -.5in
\textwidth=6.5in
\itemsep=0in
\parsep=0in
\newsectionwidth{1in}
%\usepackage{sectsty}
\usepackage[pdftex]{graphicx}
\usepackage{enumitem}
\usepackage{wrapfig}
\usepackage{helvet}
\usepackage[colorlinks = true,
            linkcolor = blue,
            urlcolor  = blue,
            citecolor = blue,
            anchorcolor = blue]{hyperref}
\setlength{\textwidth}{6.5in} % Text width of the document
\setlength{\textheight}{720pt}


\usepackage{booktabs}

\usepackage{graphicx}
\usepackage{array}
\begin{document}
\begin{center}
    \hspace{-\hoffset}
    \huge\bf{\href{}{Harshavardhan Unnibhavi}}
\end{center}
\vspace{-7mm}
\moveleft\hoffset\vbox{\hrule width 19cm height 0.5pt}
\vspace{-7mm}

\begin{center}
    \hspace{-\hoffset}
    ~\textbullet~ \(+91\) 8792812049/7292828540 ~\textbullet~ \#C327, Amber Hostel, IIT Dhanbad,Jharkhand, India~\textbullet~\#19LG HALLI,RMV 2 Stage,Bengaluru-560094(Permanent Address)    
    ~\textbullet~\href{mailto:harshavardhanu.15JE001765@ece.ism.ac.in}{harshavardhanu.15JE001765@ece.ism.ac.in}\\
    ~\textbullet~\href{https://github.com/harshanavkis}{Github}
\end{center}
\vspace{-7mm}

\begin{resume}

\section{\textbf{Education}}

\textbf{Indian Institute of Technology Dhanbad,Jharkhand,India} \hfill\textit{ July 2015 - present}\\
Bachelor of Technology in Electronics and Communication Engineering\\
Current GPA:8.38/10.00\\ \\
\textbf{National Public School,Rajajinagar,Bengaluru}\hfill\textit{July 2013 - March 2015}\\
12th CBSE board\\
Total Percentage:95\%\\ \\
\textbf{Navkis Educational Centre,Bengaluru}\hfill\textit{July 2003 - March 2013}\\
10th CBSE board\\
CGPA - 9.8\\

\section{\textbf{Research Interests}}
\begin{itemize}
\item Deep Learning
\item Computer Vision
\item Natural Language Processing
\item Robotics
\end{itemize}

\section{\textbf{Research Projects}}
\begin{itemize}
\item \textbf{Study of Properties of Wavelet transformed image and it's implementation} \hfill\textit{January 2017- April 2017}\\
~\textbullet~ Implemented wavelet transform for images in MATLAB\\
~\textbullet~ Studied the EZW algorithm for image compression 

\item \textbf{Study of optical waveguides and optical interconnects for high performance computing} \hfill\textit{August 2016 - November 2016}\\
~\textbullet~ Studied about Silicon wire and rib waveguides for electronic and photonic convergence and various methods to reduce power loss during transmission of the electromagnetic wave through the waveguide.\\
\end{itemize}

\section{\textbf{Other Projects}}
\begin{itemize}
\item \textbf{Cats Vs Dogs}\hfill\textit{August 2017 - Present}\\
~\textbullet~Created a 6 layer Convolutional Neural Network in Tensorflow.\\
~\textbullet~Each layer consists of a Convolution, Non-Linearity and Max Pooling.\\
~\textbullet~This network was trained on the Cats Vs Dogs dataset found on Kaggle.\\
~\textbullet~Achieved an accuracy of 87.5\% on the test set and 74.6\% on the validation set.\\
~\textbullet~The project can be found at this \href{https://github.com/harshanavkis/Dogs-Vs-Cats}{link}.\\
\end{itemize}

\begin{itemize}
\item \textbf{Sentiment Analysis on Rotten Tomatoes dataset}\hfill\textit{April 2017- May 2017}\\
~\textbullet~Converted the training and test dataset into word2vec representation.\\
~\textbullet~Applied KMeans clustering to find semantically related clusters.\\
~\textbullet~Trained a random forest classifier to produce the predictions.\\
~\textbullet~The project can be found at this \href{https://github.com/harshanavkis/Kaggle-Rotten-Tomatoes-Competition}{link}.\\
\end{itemize}

\begin{itemize}
\item \textbf{Robot controlled by an AVR ATMega8 Microcontroller}\hfill\textit{October 2015}\\
~\textbullet~Built a Line follower,Edge and Wall Avoider bot\\
~\textbullet~Built a GSM controlled bot using Dual Tone Multiple Frequency(DTMF) signalling
\end{itemize}

\section{\textbf{Technical Skills}}
\begin{tabular}{{l}ccccc}

\textbf{Programming languages}&\multicolumn{3}{l}{C,Python,Java,C++}\\ 
\textbf{Software Skills}&\multicolumn{3}{l}{MATLAB,RSoft,AVR Studio,OpenCV}\\
\textbf{Tools}&\multicolumn{3}{l}{Git,\LaTeX}\\
\textbf{Operating Systems}&\multicolumn{3}{l}{Windows,Linux}\\
\textbf{Libraries}&\multicolumn{3}{l}{TensorFlow,scikit-learn,numpy}\\
\textbf{Hardware Skills}&\multicolumn{3}{l}{Arduino, AVR ATMega8, 8085, TMS320C31 DSK, 31 DSK}
\end{tabular}

\section{\textbf{Relevant Courses}}
\begin{itemize}
\item Network Theory and Filter Design,Digital Circuits,Signals and Systems,Digital Signal Processing, Microprocessors
\item Data Structures,Algorithm Design and Analysis,Linear Algebra,Multivariable Calculus,Vector Calculus,Numerical and Statistical methods
\item Machine Learning(by Andrew Ng,Coursera),cs231n(Fei Fei Li and Andrej Karpathy) 
\end{itemize}


\section{\textbf{Selected Achievements}}
\begin{itemize}
\item Kishore Vaigyanik Protsahan Yojana (KVPY) Fellow(since 2015),conducted by Indian Institute of Science,Bengaluru 
\item National Talent Search Scholarship holder, NCERT, India(since 2013)
\end{itemize}


\section{\textbf{Languages known}}
English,Kannada,Hindi




\end{resume}
\end{document}