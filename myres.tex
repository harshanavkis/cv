\documentclass[margin, centered]{res}
\topmargin=-0.5in
\oddsidemargin -.5in
\evensidemargin -.5in
\textwidth=6.5in
\itemsep=0in
\parsep=0in
\newsectionwidth{1in}
%\usepackage{sectsty}
\usepackage[pdftex]{graphicx}
\usepackage{enumitem}
\usepackage{wrapfig}
\usepackage{helvet}
\usepackage[colorlinks = true,
            linkcolor = blue,
            urlcolor  = blue,
            citecolor = blue,
            anchorcolor = blue]{hyperref}
\setlength{\textwidth}{6.5in} % Text width of the document
\setlength{\textheight}{720pt}


\usepackage{booktabs}

\usepackage{graphicx}
\usepackage{array}
\begin{document}
\begin{center}
    \hspace{-\hoffset}
    \huge\bf{\href{}{Harshavardhan Unnibhavi}}
\end{center}
\vspace{-4mm}
\moveleft\hoffset\vbox{\hrule width 19cm height 0.5pt}
\vspace{-7mm}

\begin{center}
    \hspace{-\hoffset}
    ~\textbullet~\href{mailto:harshanavkis@gmail.com}{harshanavkis@gmail.com}
    ~\textbullet~\href{https://github.com/harshanavkis}{https://github.com/harshanavkis}
\end{center}
\vspace{-7mm}

\begin{resume}

\section{\textbf{Education}}

\textbf{Indian Institute of Technology Dhanbad,Jharkhand,India} \hfill\textit{ July 2015 - present}\\
B.Tech in Electronics and Communication Engineering, Minor in Computer Science\\
Current GPA:8.28/10.00\\ \\
\textbf{National Public School,Rajajinagar,Bengaluru}\hfill\textit{July 2013 - March 2015}\\
12th CBSE board\\
Total Percentage:95\%

\section{\textbf{Research Interests}}
\begin{itemize}
\item Deep Learning, Computer Vision, Robotics, Human-Computer Interaction
\end{itemize}

\section{\textbf{Internship Experience}}
\begin{itemize}
\item \textbf{Research Intern}, \textit{Simon Fraser University, Burnaby, BC, Canada}
\hfill\textit{May 2018- present}\\
~\textbullet~ Worked on Deep Feature Compression for Collaborative Intelligence\\
~\textbullet~ Developed a simulator using the Keras framework to simulate the transmission of deep features.\\
~\textbullet~ The simulator can be found at, \href{https://github.com/SFU-Multimedia-Lab/DFTS}{DFTS}.
\end{itemize}
\begin{itemize}
\item \textbf{AI and Reasoning Engine Intern}, \href{https://www.voyazer.net/}{ZapMyTrip Travel Solutions}\hfill\textit{November 2017- January 2018}\\
~\textbullet~Responsible for integrating NLP algorithms to extract contextual and factual information from travel reviews, news articles, into their application.
\end{itemize}

\section{\textbf{Publications}}
\begin{itemize}
\item H. Unnibhavi, H. Choi, S. R. Alvar, and I. V. Bajic, "DFTS: Deep Feature Transmission Simulator," IEEE Multimedia Signal Processing Workshop (MMSP), Vancouver, BC, Aug. 2018
\end{itemize}

\section{\textbf{Projects: code can be found at the GitHub link}}
\begin{itemize}
\item \textbf{EasyTorch}\hfill\textit{June 2018- Present}\\
~\textbullet~ A graphical user interface to PyTorch.\\
~\textbullet~ The aim is to create and train/test models by providing a drag and drop interface to the application, leading to faster prototyping, thus accelerating research. 
\end{itemize}


\begin{itemize}
\item \textbf{End-To-End Text to Speech Model}\hfill\textit{October 2017- December 2017}\\
~\textbullet~ Design and Implementation of a Text to Speech model in PyTorch using Deep Learning.\\
~\textbullet~ It was designed for the Hindi language.
\end{itemize}

\begin{itemize}
\item \textbf{Cats Vs Dogs}\hfill\textit{August 2017}\\
~\textbullet~Created a 6 layer Convolutional Neural Network in Tensorflow.\\
~\textbullet~Each layer consists of a Convolution, Non-Linearity and Max Pooling.\\
~\textbullet~This network was trained on the Cats Vs Dogs dataset found on Kaggle.\\
~\textbullet~Achieved an accuracy of 87.5\% on the test set.\\
~\textbullet~The project can be found at this \href{https://github.com/harshanavkis/Dogs-Vs-Cats}{link}.
\end{itemize}

\begin{itemize}
\item \textbf{Sentiment Analysis on Rotten Tomatoes dataset}\hfill\textit{April 2017- May 2017}\\
~\textbullet~Converted the training and test dataset into word2vec representation.\\
~\textbullet~Applied KMeans clustering to find semantically related clusters.\\
~\textbullet~Trained a random forest classifier to produce the predictions.\\
~\textbullet~The project can be found at this \href{https://github.com/harshanavkis/Kaggle-Rotten-Tomatoes-Competition}{link}.
\end{itemize}

\begin{itemize}
\item \textbf{Robot controlled by an AVR ATMega8 Microcontroller}\hfill\textit{October 2015}\\
~\textbullet~Built a Line follower,Edge and Wall Avoider bot\\
~\textbullet~Built a GSM controlled bot using Dual Tone Multiple Frequency(DTMF) signalling
\end{itemize}

\section{\textbf{Technical Skills}}
\begin{tabular}{{l}ccccc}

\textbf{Programming languages}&\multicolumn{3}{l}{C,Python,Java,C++,8085 Assembly}\\ 
\textbf{Software Skills}&\multicolumn{3}{l}{MATLAB, AVR Studio, Git,\LaTeX}\\
\textbf{Operating Systems}&\multicolumn{3}{l}{Windows,Linux}\\
\textbf{Frameworks}&\multicolumn{3}{l}{TensorFlow, PyTorch, Keras}\\
\textbf{Hardware Skills}&\multicolumn{3}{l}{Arduino, AVR ATMega8, 8085, TMS320C31 DSK, 31 DSK}
\end{tabular}

\section{\textbf{Relevant Courses}}
\begin{itemize}
\item Signals and Systems,Digital Signal Processing, Microprocessors, Computer Networks, Computer Architecture, Operating Systems(current)
\item Data Structures,Algorithm Design and Analysis,Linear Algebra,Multivariable Calculus,Vector Calculus,Numerical and Statistical methods
\item Machine Learning(by Andrew Ng,Coursera),cs231n(Fei Fei Li and Andrej Karpathy) 
\end{itemize}


\section{\textbf{Selected Achievements}}
\begin{itemize}
\item Was awarded the Mitacs Globalink fellowship.
\item Was selected for the KVPY scholarship program from about 150,000 students, conducted by the Indian Institute of Science, Bangalore and the Department of Science and Technology, Government of India. I was declared among the top 1\% after a rigorous examination and an interview.(Since 2015)
\item Was among the top 1000, selected from a pool of 500,000 students all over the country after clearing the National Talent Search Examination, which is a national-level scholarship program in India.(Since 2013)
\end{itemize}


\section{\textbf{Languages known}}
English(Proficient),Kannada,Hindi




\end{resume}
\end{document}